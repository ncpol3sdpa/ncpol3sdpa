\documentclass[12pt,a4paper]{article}

\usepackage[utf8]{inputenc}
\usepackage[french]{babel}
\usepackage{hyperref}
\usepackage{graphicx}
\usepackage{float}
\usepackage[T1]{fontenc}
\usepackage{fancyhdr}
\usepackage[font=footnotesize,labelfont=bf]{caption}

\setcounter{secnumdepth}{0}

\pagestyle{fancy}
\fancyhf{}
\fancyhead[C]{{\small python-poly-opt}}
\fancyfoot[C]{\thepage}
\renewcommand{\headrulewidth}{0pt}

\setlength{\parskip}{1em}

\title{
  \textbf{python-poly-opt} \\
  \large{A Python library for polynomial optimization}
}
\author{
    Alain Thirion \quad
    Mathis Cros \quad
    Nazar Chernyak \\
    Thomas Debernardi \quad
    Yann RUELLAN
}
\date{2025}

\begin{document}

\maketitle

% CONSIGNES:
% Pour montrer le recul sur votre projet et son inclusion dans notre environnement et dans votre formation à l'école, nous vous demandons de produire un petit rapport, sur les enjeux sociaux et/ou environnementaux de votre projet.
% Vous pourrez aborder dans ce rapport toute information pertinente sur l'impact direct ou indirect de votre projet sur des aspects sociétaux, artistiques, légaux, environnementaux, économiques, etc. : il peut s'agir de questions liées à la confidentialité des données que vous collectez, d'une réflexion sur l'impact environnemental (positif ou non) des modèles que vous utilisez, les questions de biais ou d'usage afférentes, etc. et son but est de démontrer un recul critique sur les aspects non purement techniques du projet.
% Le rapport devra faire *entre une et deux pages*

\section{Introduction}

% Descriptif du projet et de son but

\section{Impact de la recherche}

\section{Impact environnemental}

\section{Impact sociétal}

\section{Conclusion}

\end{document}
