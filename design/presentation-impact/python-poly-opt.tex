\documentclass[11pt,a4paper]{article}


\usepackage[margin=1.4in]{geometry}
\usepackage[utf8]{inputenc}
\usepackage[french]{babel}
\usepackage{hyperref}
\usepackage{graphicx}
\usepackage{float}
\usepackage[T1]{fontenc}
\usepackage{fancyhdr}
\usepackage[font=footnotesize,labelfont=bf]{caption}

\setcounter{secnumdepth}{0}

\pagestyle{fancy}
\fancyhf{}
\fancyhead[C]{{\small python-poly-opt}}
\fancyfoot[C]{\thepage}
\renewcommand{\headrulewidth}{0pt}

% \setlength{\parskip}{1em}

\title{
  \textbf{python-poly-opt} \\
  \large{Une bibliothèque Python pour l'optimisation polynomiale}
}
\author{
    Alain Thirion \quad
    Mathis Cros \quad
    Nazar Chernyak \\
    Thomas Debernardi \quad
    Yann RUELLAN
}
\date{\today}

\begin{document}

\maketitle

% CONSIGNES:
% Pour montrer le recul sur votre projet et son inclusion dans notre environnement et dans votre formation à l'école, nous vous demandons de produire un petit rapport, sur les enjeux sociaux et/ou environnementaux de votre projet.
% Vous pourrez aborder dans ce rapport toute information pertinente sur l'impact direct ou indirect de votre projet sur des aspects sociétaux, artistiques, légaux, environnementaux, économiques, etc. : il peut s'agir de questions liées à la confidentialité des données que vous collectez, d'une réflexion sur l'impact environnemental (positif ou non) des modèles que vous utilisez, les questions de biais ou d'usage afférentes, etc. et son but est de démontrer un recul critique sur les aspects non purement techniques du projet.
% Le rapport devra faire *entre une et deux pages*

\section{Introduction}

% Descriptif du projet et de son but

Le projet python-poly-opt est une bibliothèque Python pour l'optimisation de polynômes. Il permet de résoudre des problèmes d'optimisation en utilisant des techniques de relaxation matricielle et d'algèbre linéaire. La bibliothèque est conçue pour être facile à utiliser et à intégrer dans d'autres projets Python.

La bibliothèque est développée pour être utilisée par des chercheurs, en particulier dans le domaine du calcul quantique, mais elle peut également être utilisée dans d'autres domaines tels que la théorie des graphes.

\section{Impact de la recherche}

L'optimisation polynomiale est un domaine fondamental qui influence de nombreux secteurs de recherche. Notre bibliothèque \texttt{python-poly-opt} contribue à cet écosystème scientifique de plusieurs façons :

\begin{itemize}
    \item \textbf{Accélération de la recherche en calcul quantique} : Notre bibliothèque offre des outils spécifiques pour résoudre des problèmes d'optimisation liés aux algorithmes quantiques, contribuant potentiellement aux avancées dans ce domaine émergent.

    \item \textbf{Démocratisation des outils avancés} : En proposant une bibliothèque open-source, nous rendons accessibles des techniques d'optimisation sophistiquées à des chercheurs disposant de ressources limitées. La bibliothèque peut servir d'outil pédagogique pour l'enseignement des méthodes d'optimisation polynomiale dans les cursus universitaires.

    \item \textbf{Collaboration interdisciplinaire} : La facilité d'utilisation encourage la collaboration entre mathématiciens, informaticiens, physiciens et ingénieurs autour de problèmes d'optimisation communs.

    \item \textbf{Recherche fondamentale aux impacts imprévisibles} : Notre bibliothèque s'inscrit dans le cadre de la recherche fondamentale, dont la caractéristique principale est que ses applications concrètes et ses impacts futurs sont souvent difficiles à prévoir. Comme de nombreuses avancées scientifiques majeures, les outils d'optimisation polynomiale que nous développons aujourd'hui pourraient trouver demain des applications révolutionnaires dans des domaines que nous ne pouvons pas encore anticiper.
\end{itemize}

\section{Impact environnemental}

L'optimisation polynomiale et notre bibliothèque peuvent avoir plusieurs impacts sur l'environnement :

\begin{itemize}
    \item \textbf{Efficacité énergétique des algorithmes} : Notre bibliothèque vise à implémenter des algorithmes efficaces qui minimisent les ressources de calcul nécessaires, réduisant ainsi la consommation d'énergie associée.

    \item \textbf{Réduction des doublons de code} : Notre bibliothèque est conçue pour être modulaire et réutilisable, ce qui permet de réduire la redondance dans le code et d'optimiser l'utilisation des ressources. Cela permet en outre d'avoir un code optimisé et que chaque personne n'ait pas à réinventer la roue.

    \item \textbf{Applications à des problèmes environnementaux} : Les techniques d'optimisation polynomiale peuvent être appliquées à la modélisation et à la résolution de problèmes environnementaux complexes, comme l'optimisation des réseaux d'énergie renouvelable ou la modélisation climatique.
\end{itemize}

\section{Impact sociétal}

Notre projet présente plusieurs implications sociétales importantes :

\begin{itemize}
    \item \textbf{Accessibilité de la recherche avancée} : En tant que solution open-source, nous contribuons à réduire les inégalités d'accès aux outils scientifiques entre institutions et pays.

    \item \textbf{Formation et éducation} : Notre bibliothèque peut devenir un outil pédagogique pour former la prochaine génération de chercheurs et d'ingénieurs aux techniques d'optimisation et au calcul quantique.

    \item \textbf{Éthique de l'utilisation} : Nous sommes conscients que les techniques d'optimisation peuvent être appliquées à des fins diverses, incluant potentiellement des usages contraires à l'éthique.
\end{itemize}

\section{Conclusion}

Notre bibliothèque \texttt{python-poly-opt} est une contribution scientifique comportant des dimensions sociales et environnementales. Bien que les impacts précis de notre travail soient difficiles à prédire en raison de la nature fondamentale de la recherche, nous reconnaissons notre responsabilité concernant l'efficacité énergétique de nos algorithmes et les implications éthiques potentielles de leur utilisation.

Nous espérons que \texttt{python-poly-opt} contribuera à l'avancement du calcul quantique et d'autres domaines scientifiques dans le cadre d'une recherche ouverte, accessible et responsable.
\end{document}
